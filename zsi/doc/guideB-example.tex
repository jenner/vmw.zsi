\chapter{Example: WolframSearch}
\section{Code Generation from WSDL and XML Schema}

This section covers wsdl2py, the second way ZSI provides to access WSDL
services.  Given the path to a WSDL service, two files are generated, a 
'service' file and a 'types' file, that one can then use to access the
service.  As an example, we will use the search service provided by Wolfram
Research Inc.\copyright{}, \url{http://webservices.wolfram.com/wolframsearch/}, 
which provides a service for searching the popular MathWorld site, 
\url{http://mathworld.wolfram.com/}, among others.

\begin{verbatim}
wsdl2py --complexType http://webservices.wolfram.com/services/SearchServices/WolframSearch2.wsdl
\end{verbatim}

Run the above command to generate the service and type files.  wsdl2py uses
the {\it name} attribute of the {\it wsdl:service} element to name the resulting files.
In this example, the service name is {\it WolframSearchService}.  Therefore the files
{\it WolframSearchService_services.py} and {\it WolframSearchService_services_types.py}
should be generated.

The 'service' file contains locator, portType, and message classes.  
A locator instance is used to get an instance of a portType class, 
which is a remote proxy object. Message instances are sent and received 
through the methods of the portType instance.

The 'types' file contains class representations of the definitions and
declarations defined by all schema instances imported by the WSDL definition.
XML Schema attributes, wildcards, and derived types are not fully
handled.

\subsection{Example Use of Generated Code}

The following shows how to call a proxy method for {\it WolframSearch}.  It
assumes wsdl2py has already been run as shown in the section above.  The example
will be explained in greater detail below.

\begin{verbatim}
# import the generated class stubs
from WolframSearchService_client import *

# get a port proxy instance
loc = WolframSearchServiceLocator()
port = loc.getWolframSearchmyPortType()

# create a new request
req = WolframSearchRequest()
req.Options = req.new_Options()
req.Options.Query = 'newton'

# call the remote method
resp = port.WolframSearch(req)

# print results
print 'Search Time:', resp.Result.SearchTime
print 'Total Matches:', resp.Result.TotalMatches
for hit in resp.Result.Matches.Item:
    print '--', hit.Title
\end{verbatim}

Now each section of the code above will be explained.

\begin{verbatim}
from WolframSearchService_client import *
\end{verbatim}

We are primarily interested in the service locator that is imported.  The 
binding proxy and classes for all the messages are additionally imported.
Look at the {\it WolframSearchService_services.py} file for more information.

\begin{verbatim}
loc = WolframSearchServiceLocator()
port = loc.getWolframSearchmyPortType()
\end{verbatim}

Using an instance of the locator, we fetch an instance of the port proxy
which is used for invoking the remote methods provided by the service.  In
this case the default {\it location} specified in the {\it wsdlsoap:address}
element is used.  You can optionally pass a url to the port getter method to
specify an alternate location to be used.  The {\it portType} - {\it name} 
attribute is used to determine the method name to fetch a port proxy instance.
In this example, the port name is {\it WolframSearchmyPortType}, hence the 
method of the locator for fetching the proxy is {\it getWolframSearchmyPortType}.

The first step in calling {\it WolframSearch} is to create a request object
corresponding to the input message of the method.  In this case, the name of
the message is {\it WolframSearchRequest}.  A class representing this message
was imported from the service module.

\begin{verbatim}
req = WolframSearchRequest()
req.Options = req.new_Options()
req.Options.Query = 'newton'
\end{verbatim}

Once a request object is created we need to populate the instance with the
information we want to use in our request.  This is where the {\tt --complexType}
option we passed to wsdl2py will come in handy.  This caused the creation of 
functions for getting and setting elements and attributes of the type, class 
properties for each element, and convenience functions for creating new instances
of elements of complex types.  This functionality is explained in detail in 
subsection~\ref{subsubsection:complexType}.

Once the request instance is populated, calling the remote service is easy.  Using
the port proxy we call the method we are interested in.  An instance of the python
class representing the return type is returned by this call.  The \var{resp} object
can be used to introspect the result of the remote call.

\begin{verbatim}
resp = port.WolframSearch(req)
\end{verbatim}

Here we see that the response message, \var{resp}, represents type {\it WolframSearchReturn}.
This object has one element, {\it Result} which contains the search results for our
search of the keyword, {\tt newton}.

\begin{verbatim}
print 'Search Time:', resp.Result.SearchTime
...
\end{verbatim}

Refer to the wsdl for {\it WolframSearchService} for more details on the
returned information. 
