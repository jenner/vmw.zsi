\newcommand{\XMLSchema}{\citetitle[]{XML Schema Specification}}
\newcommand{\WSDL}{\citetitle[http://www.w3.org/TR/wsdl]{Web Services 
Description Language }}
\newcommand{\SOAP}{\citetitle[http://www.w3.org/TR/soap]{SOAP 1.1 Specification}}
\newcommand{\WSI}{\citetitle[http://www.ws-i.org/Profiles/BasicProfile-1.1.html]{Basic Profile (WS-Interop)}}
\newcommand{\URL}{\citetitle[]{Uniform Resource Locator}}

\newcommand{\WPY}{\program{wsdl2py }}
\newcommand{\WPYDIS}{\program{wsdl2dispatch }}
\newcommand{\WS}{\emph{Web Service}}
\newcommand{\XSD}{The XML Schema type library}
\newcommand{\DOCLIT}{\it{doc/literal}}
\newcommand{\RPCLIT}{\it{rpc/literal}}
\newcommand{\RPCENC}{\it{rpc/enc}}
\chapter{Introduction}

\ZSI{}, the Zolera SOAP Infrastructure, is a Python package that
provides an implementation of the SOAP specification, as described in
\SOAP.

This guide demonstrates how to use ZSI to develop \WS{} applications from a
\WSDL document.

This document is primarily concerned with demonstrating and documenting how to
use a \WS{} by creating and accessing Python data for the purposes of
sending and receiving SOAP messages.  Typecodes are used to marshall Python
datatypes into XML, which can be included in a SOAP Envelope.  The typecodes
are generated from information provided in the WSDL document, and generally
describe SOAP and XML Schema data types.  For a low-level treatment of
typecodes, and a description of SOAP-based processing see the ZSI manual.

\section{Acronyms and Terminology}

\begin{definitions}
\item{SOAP \newline Usually refering to the content and format of a message ultimately
sent and received by a \WS{}, see \SOAP{}}
\item{WSDL \newline A document describing a \WS{}'s interface, see
\WSDL{}}
\item{XMLSchema \newline Standard for modeling XML document
structure.  See \XMLSchema{}}
\item {schema document \newline a file containing a schema definition.}
\item {schema (instance)\newline The set of rules or components contained in the
assemblage of one or more schema documents.} 
\item{Element Declaration \newline A schema component that associates a
name with a type definition.  eg. \it{<element name="age" type="xsd:int">}, }
\item{GED \newline \it{Global Element Declaration}, an element declared at the
top-level of a schema.}
\item{ComplexType \newline The parent of all type definitions that can
specify attributes and children.}
\item{SimpleType \newline A simple data type like a string or integer.  The
\XMLSchema{} defines many built-in types. }
\item{\XSD \newline The \url{http://www.w3.org/2001/XMLSchema} namespace, which
contains definitions of various primitive types like string and integer, as well
as a compound type \it{complexType} used to create aggregate types.  
Conventionally the \emph{xsd} prefix is used to map to this schema.}
\item{\DOCLIT\newline document style with literal encoding}
\item{\RPCENC\newline rpc style with specified encoding, not compatible with
\WSI}
\item{\RPCLIT\newline rpc style with literal encoding.}
\end{definitions}

\section{Overview}
The ZSI \WS tools are for top-Down \WS development, using an existing WSDL
Document to create client and server applications (see \ref{section:NC}). A \WS, in the context of this document, exposes a WSDL Document describing the
service's interface, this document is typically available at a published URL (see
\URL).  The WSDL document defines SOAP bindings for communicating with the 
service. These bindings will be used to exchange SOAP messages, the contents of
these messages must adhere to the document structure specified by the schema. The 
schema is either included in the WSDL Document, imported by it, or represented
by the available built-in types (such as \emph{xsd:int, xsd:string, etc}).  

\subsection{soap bindings}
The two styles of SOAP bindings are \emph{rpc} and \emph{document}.  The use of
\it{literal} encoding is encouraged and the recommended way to develop new \WS
applications (see \WSI).  The SOAP \it{encoded} support is maintained for use
with old apps, and other SOAP toolkits restricted to \RPCENC development.  A
\DOCLIT service is typically described as an exchange of documents, while a
\RPCENC or \RPCLIT service is thought of in terms of remote procedure calls.
Whether this distinction of purpose is meaningful or useful is debatable.  \ZSI
supports all three types, but \RPCLIT and \DOCLIT are the focus of ongoing 
development.

\subsection{python tools}
\subsubsection{wsdl2py}
The \WPY script generates python code representing the various components
defined in a WSDL Document.  The second chapter introduces \WPY and demonstrates
how to create a client for a simple \WS from a WSDL document.
\subsubsection{wsdl2dispatch}
The \WPYDIS script should be run after running \WPY, since it will attempt to
import all it generates.  This sciprt generates a module containing a service
interface generated  from the WSDL Document. This interface is typically
subclassed and invoked through an HTTP server. 


\section{Not Covered}
\label{section:NC}
\begin{enumerate}
 \item{How to create a WSDL document}
 \item {How to write XML Schema}
 \item {Interoperability}
 \item{How to use Web Services without WSDL}
\end{enumerate}



\section{References}
\begin{enumerate}
 \item{Web services development patterns 
 \url{http://www-128.ibm.com/developerworks/websphere/library/techarticles/0511_flurry/0511_flurry.html}}
\end{enumerate}



