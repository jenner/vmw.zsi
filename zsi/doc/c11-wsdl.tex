\chapter{WSDL Services}

The \module{ZSI.wsdl} module provides client tools for using WSDL 1.1
(see \citetitle[http://www.w3.org/TR/wsdl]{WSDL 1.1 specification})
files to access services.

The \class{WSDLReader} class provides methods for loading WSDL service 
descriptions from URLs or XML data and creating a \class{WSDL} object.
\class{WSDL} instances represent WSDL service descriptions and provide 
an object model for building and working with those descriptions.

The \class{ServiceProxy} uses a \class{WSDL} instance to send
and receive messages.
Once the proxy has been created, each remote operation is exposed
as a method on the object.
The methods can be called with either positional or keyword arguments;
the argument types must be compatible with the types specified in the
WSDL description.

The following example shows a simple language translation service that makes
use of the complex type structures defined in the module BabelTypes:

\begin{verbatim}
from ZSI.wsdl import ServiceProxy
import BabelTypes

service = ServiceProxy('http://www.xmethods.net/sd/BabelFishService.wsdl',
		       typesmodule=BabelTypes)
value = service.BabelFish('en_de', 'This is a test!')
\end{verbatim}

The return value from a proxy method depends on the SOAP signature. If the 
remote service returns a single value, that value will be returned. If the 
remote service returns multiple ``out'' parameters, the return value of the 
proxy method will be a dictionary containing the out parameters indexed by 
name.  Because \class{ServiceProxy} makes use of the ZSI serialization / 
deserialization engine, complex return types are supported.  This means 
that an aggregation of primitives can be returned from or passed to a service
invocation according to any predefined hierarchical structure.


\begin{classdesc}{ServiceProxy}{wsdl,\optional{, service\optional{, port}}}
The \class{ServiceProxy} class provides calls to
web services. A WSDL description must be available for the 
service.

The \var{wsdl} argument may be either the URL of the service description 
or an existing \class{WSDL} instance. The optional \var{service} and 
\var{port} name the service and port within the WSDL description that 
should be used. If not specified, the first defined service and port 
will be used.

The following keyword arguments may be used:

\begin{tableiii}{l|c|p{30em}}{textrm}{Keyword}{Default}{Description}
\lineiii{\code{nsdict}}{\code{\{\}}}{Namespace dictionary to send in the
    SOAP \code{Envelope}}
\lineiii{\code{tracefile}}{\code{None}}{An object with a \code{write}
    method, where packet traces will be recorded.}
\end{tableiii}

A \class{ServiceProxy} instance, once instantiated, exposes callable 
methods that reflect the methods of the remote web service it 
represents. These methods may be called like normal methods, using 
*either* positional or keyword arguments (but not both).

When a method of a \class{ServiceProxy} is called with positional 
arguments, the arguments are mapped to the SOAP request based on 
the parameter order defined in the WSDL description. If keyword 
arguments are passed, the arguments will be mapped based on their 
names.

\end{classdesc}

\begin{classdesc}{WSDLReader}{}

\class{WSDLReader} instances are used to load WSDL service descriptions
from URLs, XML files or XML string data. The reader is implemented as a 
separate class to make it easy to create custom readers that implement 
caching policies or other optimizations.

The following methods are available:

\begin{methoddesc}{loadFromStream}{file}
Return a \class{WSDL} instance representing the service description 
contained in \var{file}. The \var{file} argument must be a file-like 
object.
\end{methoddesc}

\begin{methoddesc}{loadFromString}{data}
Returns a \class{WSDL} instance loaded from the XML string \var{data}.
\end{methoddesc}

\begin{methoddesc}{loadFromFile}{filename}
Returns a \class{WSDL} instance loaded from the file named by \var{filename}.
\end{methoddesc}

\begin{methoddesc}{loadFromURL}{url}
Returns a \class{WSDL} instance loaded from the given \var{url}.
\end{methoddesc}

\end{classdesc}

\section{Upper-Level WSDL objects}

The following classes describe encapsulate the upper-level objects
in a WSDL file.

Note that there are quite many classes defined here to
implement the WSDL object model. Instances of those classes are generally 
accessed and created through container objects rather than instantiated 
directly. Most of them simply implement a straightforward representation of 
the WSDL elements they represent. The interfaces of these objects are 
described in the next section.

An exception is defined for errors that occur while creating WSDL objects.

\begin{excdesc}{WSDLError}
This exception is raised when errors occur in the parsing or building of 
WSDL objects, usually indicating invalid structure or usage.
It is a subtype of Python's \exception{Exception} class.
\end{excdesc}

\begin{classdesc}{WSDL}{}

\class{WSDL} instances implement the WSDL object model. They are
created by loading an XML source into a \class{WSDLReader} object.

A \class{WSDL} object provides access to all of the structures that make 
up a web service description. The various ``collections'' in the WSDL 
object model (services, bindings, portTypes, etc.) are implemented as 
\class{Collection} objects that behave like ordered mappings.

The following attributes are read-only:

\begin{memberdesc}{name}
The name of the service description (associated with the \emph{definitions} 
element), or \code{None} if not specified.
\end{memberdesc}

\begin{memberdesc}{targetNamespace}
The target namespace associated with the service description, or 
\code{None} if not specified.
\end{memberdesc}

\begin{memberdesc}{documentation}
The documentation associated with the \emph{definitions} element of the 
service description, or the empty string if not specified.
\end{memberdesc}

\begin{memberdesc}{location}
The URL from which the service description was loaded, or \code{None} 
if the description was not loaded from a URL.
\end{memberdesc}

\begin{memberdesc}{services}
A collection that contains \class{Service} objects that represent the 
services that appear in the service description. The items of this 
collection may be indexed by name or ordinal. 
\end{memberdesc}

\begin{memberdesc}{messages}
A collection that contains \class{Message} objects that represent the 
messages that appear in the service description. The items of this 
collection may be indexed by name or ordinal. 
\end{memberdesc}

\begin{memberdesc}{portTypes}
A collection that contains \class{PortType} objects that represent the 
portTypes that appear in the service description. The items of this 
collection may be indexed by name or ordinal. 
\end{memberdesc}

\begin{memberdesc}{bindings}
A collection that contains \class{Binding} objects that represent the 
bindings that appear in the service description. The items of this 
collection may be indexed by name or ordinal. 
\end{memberdesc}

\begin{memberdesc}{imports}
A collection that contains \class{ImportElement} objects that represent the 
import elements that appear in the service description. The items of this 
collection may be indexed by ordinal or the target namespace URI of the 
import element.
\end{memberdesc}

\begin{memberdesc}{types}
A \class{Types} instance that contains \class{XMLSchema} objects that 
represent the schemas defined or imported by the WSDL description. The 
\class{Types} object may be indexed by ordinal or by targetNamespace to 
lookup schema objects.
\end{memberdesc}

\begin{memberdesc}{extensions}
A sequence of objects that represent WSDL \emph{extension elements}. These 
objects may be instances of classes that represent WSDL-defined extensions 
(\class{SoapBinding}, \class{SoapBodyBinding}, etc.), or DOM \class{Element} 
objects for unknown extensions.
\end{memberdesc}

\end{classdesc}

\begin{classdesc}{Service}{}
A \class{Service} object represents a WSDL \code{<service>} element.

The following attributes are read-only:

\begin{memberdesc}{name}
The name of the service.
\end{memberdesc}

\begin{memberdesc}{documentation}
The documentation associated with the element, or an empty string.
\end{memberdesc}

\begin{memberdesc}{ports}
A collection that contains \class{Port} objects that represent the 
ports defined by the service. The items of this collection may be indexed 
by name or ordinal. 
\end{memberdesc}

\begin{memberdesc}{extensions}
A sequence of any contained WSDL extensions.
\end{memberdesc}

The following method is available:

\begin{methoddesc}{getWSDL}{}
Return the parent \class{WSDL} instance of the object.
\end{methoddesc}

\end{classdesc}


\begin{classdesc}{Port}{}

A \class{Port} object represents a WSDL \code{<port>} element.

The following attributes are read-only:

\begin{memberdesc}{name}
The name of the port.
\end{memberdesc}

\begin{memberdesc}{documentation}
The documentation associated with the element, or an empty string.
\end{memberdesc}

\begin{memberdesc}{binding}
The name of the binding associated with the port.
\end{memberdesc}

\begin{memberdesc}{extensions}
A sequence of any contained WSDL extensions.
\end{memberdesc}

The following methods are available:

\begin{methoddesc}{getAddressBinding}{}
A convenience method that returns the address binding extension for the 
port, either a \class{SoapAddressBinding} or \class{HttpAddressBinding}.
Raises \exception{WSDLError} if no address binding is found.
\end{methoddesc}

\begin{methoddesc}{getService}{}
Return the parent \class{Service} instance of the object.
\end{methoddesc}

\begin{methoddesc}{getBinding}{}
Return the \class{Binding} instance associated with the port.
\end{methoddesc}

\begin{methoddesc}{getPortType}{}
Return the \class{PortType} instance associated with the port.
\end{methoddesc}

\end{classdesc}

\begin{classdesc}{PortType}{}

A \class{PortType} object represents a WSDL \code{<portType>} element.

The following attributes are read-only:

\begin{memberdesc}{name}
The name of the portType.
\end{memberdesc}

\begin{memberdesc}{documentation}
The documentation associated with the element, or an empty string.
\end{memberdesc}

\begin{memberdesc}{operations}
A collection that contains \class{Operation} objects that represent the 
operations in the portType. The items of this collection may be indexed 
by name or ordinal. 
\end{memberdesc}

The following method is available:

\begin{methoddesc}{getWSDL}{}
Return the parent \class{WSDL} instance of the object.
\end{methoddesc}

\end{classdesc}

\begin{classdesc}{Operation}{}

A \class{Operation} object represents a WSDL \code{<operation>} element 
within a \code{portType} element.

The following attributes are read-only:

\begin{memberdesc}{name}
The name of the operation.
\end{memberdesc}

\begin{memberdesc}{documentation}
The documentation associated with the element, or an empty string.
\end{memberdesc}

\begin{memberdesc}{parameterOrder}
A string representing the \code{parameterOrder} attribute of the operation, 
or \code{None} if the attribute is not defined.
\end{memberdesc}

\begin{memberdesc}{input}
A \class{MessageRole} instance representing the \code{<input>} element of 
the operation binding, or \code{None} if no input element is present.
\end{memberdesc}

\begin{memberdesc}{output}
A \class{MessageRole} instance representing the \code{<output>} element of 
the operation, or \code{None} if no output element is present.
\end{memberdesc}

\begin{memberdesc}{faults}
A collection of \class{MessageRole} instances representing the \code{<fault>} 
elements of the operation.
\end{memberdesc}

The following methods are available:

\begin{methoddesc}{getPortType}{}
Return the parent \class{PortType} instance of the object.
\end{methoddesc}

\begin{methoddesc}{getInputMessage}{}
Return \class{Message} object associated with the input to the operation.
\end{methoddesc}

\begin{methoddesc}{getOutputMessage}{}
Return \class{Message} object associated with the output of the operation.
\end{methoddesc}

\begin{methoddesc}{getFaultMessage}{name}
Return \class{Message} object associated with the named fault.
\end{methoddesc}

\end{classdesc}

\begin{classdesc}{MessageRole}{}

\class{MessageRole} objects represent WSDL \code{<input>}, \code{<output>} 
and \code{<fault>} elements within an operation.

The following attributes are read-only:

\begin{memberdesc}{name}
The name attribute of the element.
\end{memberdesc}

\begin{memberdesc}{type}
The type of the element, one of \code{'input'}, \code{'output'} or 
\code{'fault'}.
\end{memberdesc}

\begin{memberdesc}{message}
The name of the message associated with the object.
\end{memberdesc}

\begin{memberdesc}{documentation}
The documentation associated with the element, or an empty string.
\end{memberdesc}

\end{classdesc}

\begin{classdesc}{Binding}{}

A \class{Binding} object represents a WSDL \code{<binding>} element.

The following attributes are read-only:

\begin{memberdesc}{name}
The name of the binding.
\end{memberdesc}

\begin{memberdesc}{documentation}
The documentation associated with the element, or an empty string.
\end{memberdesc}

\begin{memberdesc}{type}
The name of the portType the binding is associated with.
\end{memberdesc}

\begin{memberdesc}{operations}
A collection that contains \class{OperationBinding} objects that represent 
the contained operation bindings.
\end{memberdesc}

\begin{memberdesc}{extensions}
A sequence of any contained WSDL extensions.
\end{memberdesc}

The following methods are available:

\begin{methoddesc}{getWSDL}{}
Return the parent \class{WSDL} instance of the object.
\end{methoddesc}

\begin{methoddesc}{getPortType}{}
Return the \class{PortType} object associated with the binding.
\end{methoddesc}

\begin{methoddesc}{findBinding}{kind}
Find a binding extension in the binding. The \var{kind} can be a class 
object if the wanted extension is one of the WSDL-defined types (such as 
\class{SoapBinding} or \class{HttpBinding}). If the extension is not one of 
the supported types, \var{kind} can be a tuple of the form 
\code{(namespace-URI, localname)}, which will be used to try to find a 
matching DOM \class{Element}.
\end{methoddesc}

\begin{methoddesc}{findBindings}{kind}
The same as \method{findBinding()}, but will return multiple values of 
the given \var{kind}.
\end{methoddesc}

\end{classdesc}

\begin{classdesc}{OperationBinding}{}

A \class{OperationBinding} object represents a WSDL \code{<operation>} 
element within a binding element.

The following attributes are read-only:

\begin{memberdesc}{name}
The name of the operation binding.
\end{memberdesc}

\begin{memberdesc}{documentation}
The documentation associated with the element, or an empty string.
\end{memberdesc}

\begin{memberdesc}{input}
A \class{MessageRoleBinding} instance representing the \code{<input>} 
element of the operation binding, or \code{None} if no input element is 
present.
\end{memberdesc}

\begin{memberdesc}{output}
A \class{MessageRoleBinding} instance representing the \code{<output>} 
element of the operation binding, or \code{None} if no output element is 
present.
\end{memberdesc}

\begin{memberdesc}{faults}
A collection of \class{MessageRoleBinding} instances representing the 
\code{<fault>} elements of the operation binding.
\end{memberdesc}

\begin{memberdesc}{extensions}
A sequence of any contained WSDL extensions.
\end{memberdesc}

The following methods are available:

\begin{methoddesc}{getBinding}{}
Return the parent \class{Binding} instance of the operation binding.
\end{methoddesc}

\begin{methoddesc}{getOperation}{}
Return the abstract \class{Operation} associated with the operation binding.
\end{methoddesc}

\begin{methoddesc}{findBinding}{kind}
Find a binding extension in the operation binding. The \var{kind} can be a 
class object if the wanted extension is one of the WSDL-defined types (such 
as \class{SoapOperationsBinding} or \class{HttpOperationBinding}). 

If the extension is not one of the supported types, \var{kind} can be a
tuple of the form \code{(namespace-URI, localname)}, which will be used to 
try to find a matching DOM \class{Element}.
\end{methoddesc}

\begin{methoddesc}{findBindings}{kind}
The same as \method{findBinding()}, but will return multiple values of 
the given \var{kind}.
\end{methoddesc}

\end{classdesc}


\begin{classdesc}{MessageRoleBinding}{}

\class{MessageRoleBinding} objects represent WSDL \code{<input>}, 
\code{<output>} and \code{<fault>} elements within an operation binding.

The following attributes are read-only:

\begin{memberdesc}{name}
The name attribute of the element, for fault elements. This is always 
\code{None} for input and output elements.
\end{memberdesc}

\begin{memberdesc}{type}
The type of the element, one of \code{'input'}, \code{'output'} or 
\code{'fault'}.
\end{memberdesc}

\begin{memberdesc}{documentation}
The documentation associated with the element, or an empty string.
\end{memberdesc}

\begin{memberdesc}{extensions}
A sequence of any contained WSDL extensions.
\end{memberdesc}

The following methods are available:

\begin{methoddesc}{findBinding}{kind}
Find a binding extension in the message role binding. The \var{kind} can be a 
class object if the wanted extension is one of the WSDL-defined types.

If the extension is not one of the supported types, \var{kind} can be a
tuple of the form \code{(namespace-URI, localname)}, which will be used to 
try to find a matching DOM \class{Element}.
\end{methoddesc}

\begin{methoddesc}{findBindings}{kind}
The same as \method{findBinding()}, but will return multiple values of 
the given \var{kind}.
\end{methoddesc}

\end{classdesc}


\begin{classdesc}{Message}{}

A \class{Message} object represents a WSDL \code{<message>} element.

The following attributes are read-only:

\begin{memberdesc}{name}
The name of the message.
\end{memberdesc}

\begin{memberdesc}{documentation}
The documentation associated with the element, or an empty string.
\end{memberdesc}

\begin{memberdesc}{parts}
A collection that contains \class{MessagePart} objects that represent the 
parts of the message. The items of this collection may be indexed 
by name or ordinal. 
\end{memberdesc}
\end{classdesc}


\begin{classdesc}{MessagePart}{}

A \class{MessagePart} object represents a WSDL \code{<part>} element.

The following attributes are read-only:

\begin{memberdesc}{name}
The name of the message part.
\end{memberdesc}

\begin{memberdesc}{documentation}
The documentation associated with the element, or an empty string.
\end{memberdesc}

\begin{memberdesc}{type}
A tuple of the form \code{(namespace-URI, localname)}, or \code{None} 
if the \member{type} attribute is not defined.
\end{memberdesc}

\begin{memberdesc}{element}
A tuple of the form \code{(namespace-URI, localname)}, or \code{None} 
if the \member{element} attribute is not defined.
\end{memberdesc}
\end{classdesc}


\begin{classdesc}{Types}{}

The following attributes are read-only:

A \class{Types} object represents a WSDL \code{<types>} element. It acts 
as an ordered collection containing \class{XMLSchema} instances associated 
with the service description (either directly defined in a \code{<types>} 
element, or included via import). The \class{Types} object can be indexed 
by ordinal or by the \code{targetNamespace} of the contained schemas.

\begin{memberdesc}{documentation}
The documentation associated with the element, or an empty string.
\end{memberdesc}

\begin{memberdesc}{extensions}
A sequence of any contained WSDL extensions.
\end{memberdesc}

The following method is available:

\begin{methoddesc}{getWSDL}{}
Return the parent \class{WSDL} instance of the object.
\end{methoddesc}
\end{classdesc}

\begin{classdesc}{ImportElement}{}

A \class{ImportElement} object represents a WSDL \code{<import>} element.

The following attributes are read-only:

\begin{memberdesc}{namespace}
The namespace attribute of the import element.
\end{memberdesc}

The following method is available:

\begin{memberdesc}{location}
The location attribute of the import element.
\end{memberdesc}
\end{classdesc}


\section{Binding Classes}

The \module{WSDLTools} module contains a number of classes that represent 
the binding extensions defined in the WSDL specification. These classes 
are straightforward, reflecting the attributes of the corresponding 
XML elements, so they are not documented exhaustively here.

\begin{classdesc}{SoapBinding}{transport\optional{, style}}
Represents a \code{<soap:binding>} element.
\end{classdesc}

\begin{classdesc}{SoapAddressBinding}{location}
Represents a \code{<soap:address>} element.
\end{classdesc}

\begin{classdesc}{SoapOperationBinding}{}
Represents a \code{<soap:operation>} element.
\end{classdesc}

\begin{classdesc}{SoapBodyBinding}{}
Represents a \code{<soap:body>} element.
\end{classdesc}

\begin{classdesc}{SoapFaultBinding}{}
Represents a \code{<soap:fault>} element.
\end{classdesc}

\begin{classdesc}{SoapHeaderBinding}{}
Represents a \code{<soap:header>} element.
\end{classdesc}

\begin{classdesc}{SoapHeaderFaultBinding}{}
Represents a \code{<soap:headerfault>} element.
\end{classdesc}

\begin{classdesc}{HttpBinding}{}
Represents a \code{<http:binding>} element.
\end{classdesc}

\begin{classdesc}{HttpAddressBinding}{}
Represents a \code{<http:address>} element.
\end{classdesc}

\begin{classdesc}{HttpOperationBinding}{}
Represents a \code{<http:operation>} element.
\end{classdesc}

\begin{classdesc}{HttpUrlReplacementBinding}{}
Represents a \code{<http:urlReplacement>} element.
\end{classdesc}

\begin{classdesc}{HttpUrlEncodedBinding}{}
Represents a \code{<http:urlEncoded>} element.
\end{classdesc}

\begin{classdesc}{MimeMultipartRelatedBinding}{}
Represents a \code{<mime:multipartRelated>} element.
\end{classdesc}

\begin{classdesc}{MimePartBinding}{}
Represents a \code{<mime:part>} element.
\end{classdesc}

\begin{classdesc}{MimeContentBinding}{}
Represents a \code{<mime:content>} element.
\end{classdesc}

\begin{classdesc}{MimeXmlBinding}{}
Represents a \code{<mime:mimeXml>} element.
\end{classdesc}
