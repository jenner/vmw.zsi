\chapter{wsdl2py scrpt}

\section{Command Line Flags}

\subsection{General Flags}
\begin{description}
\item[-h, ---help] Display the help message and available command line
flags that can be passed to wsdl2py.
\item[-f FILE, ---file=FILE] Create bindings for the WSDL which is located at
the local file path.
\item[-u URL, ---url=URL] Create bindings for the remote WSDL which is located
at the provided URL.
\item[-x, ---schema] Just process a schema (xsd) file and generate the types
mapping file.
\item[-d, ---debug] Output verbose debugging messages during code generation.
\item[-o OUTPUT_DIR, ---output-dir=OUTPUT_DIR] Write generated files to OUTPUT_DIR.
\end{description}

\subsection{Typecode Extensions (Stable) }
\begin{description}
\item[-b, ---complexType (more in section~\label{subsubsection:complexType})]
Generate convenience functions for complexTypes.  This includes getters,
setters, factory methods, and properties.  ** Do NOT use with --simple-naming **
\end{description}

\subsection{Development Extensions (Unstable) }
\begin{description}
\item[-a, ---address] WS-Addressing support.  The WS-Addressing schema must be
included in the corresponding WSDL.
\item[-w, ---twisted] Generate a twisted.web client.  Dependencies: 
python\verb!>=!2.4, Twisted\verb!>=!2.0.0, TwistedWeb\verb!>=!0.5.0
\end{description}

\subsection{Customizations (Unstable) }
\begin{description}
\item[-e, ---extended] Do extended code generation.
\item[-z ANAME, ---aname=ANAME] Use a custom function, ANAME, for attribute name
creation.
\item[-t TYPES, ---types=TYPES] Dump the generated type mappings to a file
named, ``TYPES.py''.
\item[-s, ---simple-naming] Simplify the generated naming.
\item[-c CLIENTCLASSSUFFIX, ---clientClassSuffix=CLIENTCLASSSUFFIX] The suffic
to use for service client class. (default ``SOAP'')
\item[-m PYCLASSMAPMODULE, ---pyclassMapModule=PYCLASSMAPMODULE] Use the
existing existing type mapping file to determine the ``pyclass'' objects to be
used.  The module should contain an attribute, ``mapping'', which is a
dictionary of form, {schemaTypeName: (moduleName.py, className)}.
\end{description}
